%%
%% Copyright 2023 Christoph P. Neumann
%%
%% This LaTeX class provides a simple interface for creating
%% a verfy fancy Curriculum Vitae. At the moment only CVs in
%% the german language are supported.
%%
%% This file is free property; as a special exception the author
%% gives unlimited permission to copy and/or distribute it, with
%% or without modifications, as long as this notice is
%% preserved.
%%
%% This file is distributed in the hope that it will be useful,
%% but WITHOUT ANY WARRANTY, to the extent permitted by law;
%% without even the implied warranty of MERCHANTABILITY or
%% FITNESS FOR A PARTICULAR PURPOSE.

\documentclass{ltxdoc}

\CodelineNumbered
\EnableCrossrefs
\CodelineIndex
\RecordChanges
\GetFileInfo{neutr.cls}

\parskip1.0ex
\parindent0.0ex

\begin{document}
\title{\textsf{neutr}\\
Neumann Technical Reports}
\author{Christoph P.\ Neumann \texttt{$<$cyberpetaneuron@gmail.com$>$}}
\maketitle
\PrintChanges

\begin{abstract}
The \texttt{neutr}--class provides a convenient environment
for writing technical reports.
The style is adopted from IEEE.
The neutr class requires the latex/pdflatex
in combination with biblatex using a biber backend..
\end{abstract}

\section{Installation}

The \texttt{zip} or \texttt{tar.gz} file comes with a \texttt{neutr.ins}
and a \texttt{neutr.dtx} file included which contains the \LaTeX\ stuff.

To extract the class files call:

\begin{verbatim}
  $ latex neutr.ins
\end{verbatim}

This call will extract all \LaTeX\ specific files to the current
directory. You can either use the files for a single
cv project or you can integrate the files into your \TeX\ installation.

If you just want to use \textsf{neutr} for a single curriculum vitae
project, the simplest way is just to copy the generated files to the
folder of the project.

If you want to integrate \textsf{neutr} into  your \TeX\ installation,
create a directory \texttt{tex/latex/neutr} beneath your \TeX\ installation
(e.g.~beneath \texttt{/usr/share/texmf}) and copy all files from the
current directory there. Now call:

\begin{verbatim}
  $ mktexlsr
\end{verbatim}

to update the file--cache of \LaTeX.

Hint: The \textsf{neutr} distribution contains a sample docstrip configuration
in \texttt{docstrip.cfg} via which files can be distributed
automatically to their correct positions inside a \LaTeX\ installation.
Feel free to adapt this file to your environment and afterwards call
\texttt{latex neutr.ins} to install the package to its right place.

\section{Templates}

For a quick start the \textsf{iaria-lite} distribution contains document
templates.
The templates can be found in the \texttt{neutr-example-neumann.zip} file.


\section{Documentclass}

\DescribeMacro{documentclass neutr}
This package provides the documentclass \texttt{neutr}. The documentclass
supports the following options:


\begin{itemize}
\item |conference| Passed to IEEEtran
\item |a4paper|    Passed to IEEEtran
\item |flushend|   Activate flushend package
\item |pbalance|   Activate pbalance package
\end{itemize}


\section{Requirements}

We instrument several other \LaTeX\ packages for different purposes,
which must be available under your installation.

\begin{itemize}
\item IEEEtran
\item acronym
\item afterpage
\item babel
\item biblatex
\item booktabs
\item ccicons
\item cleveref
\item csquotes
\item diagbox
\item etoolbox
\item extdash
\item flushend
\item fontenc
\item graphicx
\item hypcap
\item hyperref
\item hyperref
\item inputenc
\item lipsum
\item listings
\item lmodern
\item lscape
\item microtype
\item newtxtt
\item orcidlink
\item paralist
\item pbalance
\item pdfcomment
\item pdflscape
\item siunitx
\item stfloats
\item subfig
\item tcolorbox
\item times
\item upquote
\item url
\item xcolor
\item xpatch
\item xspace
\end{itemize}

\end{document}